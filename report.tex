%FILL THESE IN
\def\mytitle{Computer Graphics \\ Course work Part 1 - Report}
\def\mykeywords{3D scene, OpenGL, C++, GLSL, lighting, shadows, normal mapping, real time, phong}
\def\myauthor{Dimitar Hristov}
\def\contact{40201757@live.napier.ac.uk}
\def\mymodule{Computer Graphics (SET08116)}
%YOU DON'T NEED TO TOUCH ANYTHING BELOW
\documentclass[10pt, a4paper]{article}
\usepackage[a4paper,outer=1.5cm,inner=1.5cm,top=1.75cm,bottom=1.5cm]{geometry}
\twocolumn
\usepackage{graphicx}
\graphicspath{{./images/}}
%colour our links, remove weird boxes
\usepackage[colorlinks,linkcolor={black},citecolor={blue!80!black},urlcolor={blue!80!black}]{hyperref}
%Stop indentation on new paragraphs
\usepackage[parfill]{parskip}
%% all this is for Arial
\usepackage[english]{babel}
\usepackage[T1]{fontenc}
\usepackage{uarial}
\renewcommand{\familydefault}{\sfdefault}
%Napier logo top right
\usepackage{watermark}
%Lorem Ipusm dolor please don't leave any in you final repot ;)
\usepackage{lipsum}
\usepackage{xcolor}
\usepackage{listings}
%give us the Capital H that we all know and love
\usepackage{float}
%tone down the linespacing after section titles
\usepackage{titlesec}
%Cool maths printing
\usepackage{amsmath}
%PseudoCode
\usepackage{algorithm2e}

\titlespacing{\subsection}{0pt}{\parskip}{-3pt}
\titlespacing{\subsubsection}{0pt}{\parskip}{-\parskip}
\titlespacing{\paragraph}{0pt}{\parskip}{\parskip}
\newcommand{\figuremacro}[5]{
    \begin{figure}[#1]
        \centering
        \includegraphics[width=#5\columnwidth]{#2}
        \caption[#3]{\textbf{#3}#4}
        \label{fig:#2}
    \end{figure}
}

\lstset{
	escapeinside={/*@}{@*/}, language=C++,
	basicstyle=\fontsize{8.5}{12}\selectfont,
	numbers=left,numbersep=2pt,xleftmargin=2pt,frame=tb,
    columns=fullflexible,showstringspaces=false,tabsize=4,
    keepspaces=true,showtabs=false,showspaces=false,
    backgroundcolor=\color{white}, morekeywords={inline,public,
    class,private,protected,struct},captionpos=t,lineskip=-0.4em,
	aboveskip=10pt, extendedchars=true, breaklines=true,
	prebreak = \raisebox{0ex}[0ex][0ex]{\ensuremath{\hookleftarrow}},
	keywordstyle=\color[rgb]{0,0,1},
	commentstyle=\color[rgb]{0.133,0.545,0.133},
	stringstyle=\color[rgb]{0.627,0.126,0.941}
}

\thiswatermark{\centering \put(336.5,-38.0){\includegraphics[scale=0.8]{logo}} }
\title{\mytitle}
\author{\myauthor\hspace{1em}\\\contact\\Edinburgh Napier University\hspace{0.5em}-\hspace{0.5em}\mymodule}
\date{}
\hypersetup{pdfauthor=\myauthor,pdftitle=\mytitle,pdfkeywords=\mykeywords}
\sloppy
\begin{document}
	\maketitle
	\begin{abstract}
		 The aim of this project is to create a realistic 3D scene, rendered in real-time. The project is inspired by the series \textit{Games of Thrones} and previous years projects found on the games website of Napier University. A wide variety of graphics techniques were used to create the 3D scene, from multiple lights and light types to shadowing, material shading and transform hierarchy.
	\end{abstract}
    
	\textbf{Keywords -- }{\mykeywords}
    %START FROM HERE
	\section{Introduction}
    \paragraph{Scene parts} The project is meant to be visually intriguing and more importantly it is meant to demonstrate core understandings of Computer Graphics principles. The 3D scene involves:   
    \begin{itemize}
    	\item a miniature model of the Earth and the Moon, rotating around it;
    	\item a wall and a spot light demonstrating shadows;
    	\item a realistic dragon egg made with normal mapping;
    	\item a model of a dragon next to the Earth, protecting its egg;
    	\item geometry objects moved with hierarchical transformations (the dragon egg protectors);
    	\item skybox that brings to the scene background and completeness;
    \end{itemize}
	\paragraph{Graphics effects}The graphics effects implemented in this project include:
	\begin{itemize}
		\item multiple light types (directional, spot and point light);
		\item texturing and normal mapping that give high level of details;
		\item shadows that make the scene more realistic;
	\end{itemize}
    There are two types of cameras implemented within the project: \textit{free and target camera}. The free camera allows the user to go around and explore the 3D scene and the four target cameras show the scene from four static points of view.
    
    The exact implementations of these graphics techniques are discussed later in the report.
    \\\\You should cite References like this: \cite{Keshav}. The references are saved in an external .bib file, and will automatically be added ot the bibliography at the end once cited.
    
    \figuremacro{h}{placeholder}{ImageTitle}{ - Some Descriptive Text}{1.0}
	
	\section{Related Work}
	
	All of the techniques used in this project can be found in the workbook for the Computer Graphics module - SET08116 at Edinburgh Napier University. The required skills were developed during the practical sessions of the module. Some of graphics techniques had to be taken further in order to develop the final 3D scene for this project.
	
	Some common formatting you may need uses these commands for \textbf{Bold Text}, \textit{Italics}, and \underline{underlined}.
	\subsection{LineBreaks}
	Here is a line
    
    Here is a line followed by a double line break.
	This line is only one line break down from the above, Notice that latex can ignore this
    
    We can force a break \\ with the break operator.
    
	\subsection{Maths}
    Embedding Maths is Latex's bread and butter    
    
    {\centering \Large \(
        J = \begin{bmatrix}
            \frac{\delta e}{\delta \theta _0}
            \frac{\delta e}{\delta \theta _1}
            \frac{\delta e}{\delta \theta _2}
        \end{bmatrix}
        = e_{current} - e_{target} 
    \)\par}
	
	\subsection{Code Listing}
    You can load segments of code from a file, or embed them directly.
    
\begin{lstlisting}[caption = Hello World! in c++]
#include <iostream>

int main() {
    std::cout << "Hello World!" << std::endl;
    std::cin.get();
    return 0;
}
\end{lstlisting}

\lstinputlisting[caption = Hello World! in python script]{./sourceCode/hello.py}
    
\subsection{PseudoCode}

\begin{algorithm}[h]
\For{$i = 0$ \KwTo $100$}{
 print\_number = true\;
\If{i is divisible by 3}{
 print "Fizz"\;
 print\_number = false\;
}
\If{i is divisible by 5}{
 print "Buzz"\;
 print\_number = false\;
}
\If{print\_number}{
    print i\;
}
print a newline\;
}
\caption{FizzBuzz}
\end{algorithm}
	\section{Implementation}
		
	There are a number of elements that are used together in order to make the scene alluring. These elements are:

	\subsection{Multiple lights}
	There are three types of lighting sources implemented in the project: directional, spot and point lights. The directional light is used to light the whole scene.
	\subsection{Texturing and Normal Mapping}
	Texturing and Normal Mapping
	\subsection{Shadows}
	Shadows
	\subsection{Moving objects}
	Moving objects
	\subsection{Hierarchical Transformations}
	Hierarchical Transformations
	\subsection{Skybox}
	Skybox
		
	\section{Conclusion}	
\bibliographystyle{ieeetr}
\bibliography{references}
		
\end{document}
